\documentclass[12pt,a4paper]{amsart}
% ukazi za delo s slovenscino -- izberi kodiranje, ki ti ustreza
\usepackage[slovene]{babel}
%\usepackage[cp1250]{inputenc}
\usepackage[T1]{fontenc}
\usepackage[utf8]{inputenc}
\usepackage{amsmath,amssymb,amsfonts}
\usepackage{url}
\usepackage{graphicx}
%\usepackage[demo]{graphicx}
%\usepackage[normalem]{ulem}
\usepackage[dvipsnames,usenames]{color}
\usepackage{hyperref}
\hypersetup{
     colorlinks   = true,
     citecolor    = gray
}

% ne spreminjaj podatkov, ki vplivajo na obliko strani
\textwidth 15cm
\textheight 24cm
\oddsidemargin.5cm
\evensidemargin.5cm
\topmargin-5mm
\addtolength{\footskip}{10pt}
\pagestyle{plain}
\overfullrule=15pt % oznaci predlogo vrstico


% ukazi za matematicna okolja
\theoremstyle{definition} % tekst napisan pokoncno
\newtheorem{definicija}{Definicija}[section]
\newtheorem{primer}[definicija]{Primer}
\newtheorem{opomba}[definicija]{Opomba}

\renewcommand\endprimer{\hfill$\diamondsuit$}


\theoremstyle{plain} % tekst napisan posevno
\newtheorem{lema}[definicija]{Lema}
\newtheorem{izrek}[definicija]{Izrek}
\newtheorem{trditev}[definicija]{Trditev}
\newtheorem{posledica}[definicija]{Posledica}


% za stevilske mnozice uporabi naslednje simbole
\newcommand{\R}{\mathbb R}
\newcommand{\N}{\mathbb N}
\newcommand{\Z}{\mathbb Z}
\newcommand{\C}{\mathbb C}
\newcommand{\Q}{\mathbb Q}

% ukaz za slovarsko geslo
\newlength{\odstavek}
\setlength{\odstavek}{\parindent}
\newcommand{\geslo}[2]{\noindent\textbf{#1}\hspace*{3mm}\hangindent=\parindent\hangafter=1 #2}

% naslednje ukaze ustrezno popravi
\newcommand{\program}{Finančna matematika} % ime studijskega programa: Matematika/Finan"cna matematika
\newcommand{\imeavtorja}{Luka Lodrant} % ime avtorja
\newcommand{\imementorja}{Janoš Vidali} % akademski naziv in ime mentorja
\newcommand{\naslovdela}{Razbitje grafa na dva disjunktna liha podgrafa}
\newcommand{\letnica}{2019} %letnica


% vstavi svoje definicije ...




\begin{document}

% od tod do povzetka ne spreminjaj nicesar
\thispagestyle{empty}
\noindent{\large
UNIVERZA V LJUBLJANI\\[1mm]
FAKULTETA ZA MATEMATIKO IN FIZIKO\\[5mm]
\program\ -- 1.~stopnja}
\vfill

\begin{center}{\large
\imeavtorja\\[2mm]
{\bf \naslovdela}\\[10mm]
Projekt v povezavi z OR\\[1cm]}

\end{center}
\vfill

\noindent{\large
Ljubljana, \letnica}
\pagebreak

\thispagestyle{empty}
\hypersetup{linkcolor = black}
\tableofcontents
\pagebreak

\section{Navodilo}

In the paper
https://arxiv.org/abs/1802.07991
the authors gave a characterization when a graph can be decomposed into two disjoint odd subgraphs. Their result, based on the Gaussian elimination method, implies a polynomial time algorithm to verify this kind od decomposability. Implement this algorithm; then for small values of
n, verify the percentage of graphs on n vertices that can be decomposed in the prescribed way.
Similarly, you can also consider other classes of graphs, say small bipartite graphs, trees, . . .





\section{Uvod}

Osnovni predmet obravnave tega dela bodo \emph{multigrafi}, ki lahko vzporedne povezave, nimajo pa zank. Grafu brez podvojenih povezav in zank rečemo \emph{enostaven graf}.
Naj bo $G=(V(G),  E(G))$ multigraf. Z $OddV(G)$ bomo označili množico oglišč z lihimi stopnjami, z $EvenV(G)$ pa množico oglišč s sodimi stopnjami.
Lih (oz. sod) podgraf grafa $G$ je podgraf, katerega vsa oglišča so lihe (oz. sode) stopnje.
\\

Naš cilj je, da ugotovimo, kdaj je možno graf razbiti na 2 podgrafa lihe stopnje in za to implementirati polinomski algoritem.
Z uporabo tega algoritma pa bomo v drugem delu analizirali kakšen delež posebnih kategorij grafov je možno na tak način razbiti.
\\

Projekt bomo implementirali v programskem jeziku \emph{Python3}, veliko pa bomo uporabljali knjižnico \emph{networkx} za delo z grafi.

\section{Matematična podlaga}

\section{Implementacija algoritma}

\section{Analiza posebnih grafov}

\subsection{Mali grafi}
\subsection{Dvodelni grafi}
\subsection{Drevesa in gozdovi}



\end{document}